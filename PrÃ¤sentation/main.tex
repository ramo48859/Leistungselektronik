%%% Template created by Susanne Hametner and Doris Pargfrieder

%-=-=-=-=-=-=-=-=-=-=-=-=-=-=-=-=-=-=-=-=-=-=-=-=
%
%        LOADING DOCUMENT
%
%-=-=-=-=-=-=-=-=-=-=-=-=-=-=-=-=-=-=-=-=-=-=-=-=

% Consider options 'german' and 'handout'
% For 4:3 aspect ratio, set the option 'aspectratio=43'
\documentclass[aspectratio=169,9pt]{beamer}

% -=-=-=-=-=-=-=-=-=-=-=-=-=-=-=-=-=-=-=-=-=-=-=-=-=-=-=-=-=-=
% Choose among the following options for the JKU-beamer-theme:
% -=-=-=-=-=-=-=-=-=-=-=-=-=-=-=-=-=-=-=-=-=-=-=-=-=-=-=-=-=-=
% german
% protectframetitle
% nopagenumber
% nosectionpage
% nojkuFooter
% greyText
% RE
% SOWI
% TNF
% MED
% mac
\usetheme[TNF]{jku}

%-=-=-=-=-=-=-=-=-=-=-=-=-=-=-=-=-=-=-=-=-=-=-=-=
%        LOADING PACKAGES
%-=-=-=-=-=-=-=-=-=-=-=-=-=-=-=-=-=-=-=-=-=-=-=-=
\usepackage[utf8]{inputenc}
\usepackage{transparent}
\usepackage{amsmath,bm}
\usepackage[backend=biber,style=ieee,autocite=footnote]{biblatex}
\DeclareAutoCiteCommand{footnote}{\footfullcite}{\footfullcite}
\renewcommand\multicitedelim{\addsemicolon\space}
\addbibresource{main.bib}
\usepackage{circuitikz}
\usetikzlibrary{arrows.meta,decorations.pathmorphing,decorations.pathreplacing,calligraphy,calc,fit,decorations.markings,shapes,spy,backgrounds,
external,positioning,matrix}

\tikzset{>=latex}
%\usetikzlibrary{external}
%\tikzexternalize[prefix=compiledFig/]
\usepackage{amsmath}
\usepackage{siunitx}
\usepackage{subcaption}
\pgfplotsset{compat=1.11,
        /pgfplots/ybar legend/.style={
        /pgfplots/legend image code/.code={%
        %\draw[##1,/tikz/.cd,yshift=-0.25em]
                %(0cm,0cm) rectangle (3pt,0.8em);},
        \draw[##1,/tikz/.cd,bar width=3pt,yshift=-0.2em,bar shift=0pt]
                plot coordinates {(0cm,0.8em)};},
},
}
\newcommand*\subtxt[1]{_{\textnormal{#1}}}
\DeclareRobustCommand\_{\ifmmode\expandafter\subtxt\else\textunderscore\fi}
\usepgfplotslibrary{groupplots}
\DeclareMathOperator*{\argmax}{arg\,max}
\DeclareMathOperator*{\argmin}{arg\,min}
%-=-=-=-=-=-=-=-=-=-=-=-=-=-=-=-=-=-=-=-=-=-=-=-=
%
%	PRESENTATION INFORMATION
%
%-=-=-=-=-=-=-=-=-=-=-=-=-=-=-=-=-=-=-=-=-=-=-=-=

\title{Halbbrückenansteuerung(Bootstrap)}
\subtitle{}
\author{Authors}
\institute{Institute for Communications Engineering and RF-Systems}
%\date{}
%\partnerlogo{fwf.png}

%\bibliographystyle{abbrv}

\begin{document}

%\jkulogoblack
%
%\jkulogotheme
%
%\jkulogowhite


%-=-=-=-=-=-=-=-=-=-=-=-=-=-=-=-=-=-=-=-=-=-=-=-=
%
%	TITLE PAGE
%
%-=-=-=-=-=-=-=-=-=-=-=-=-=-=-=-=-=-=-=-=-=-=-=-=

\maketitle

%-=-=-=-=-=-=-=-=-=-=-=-=-=-=-=-=-=-=-=-=-=-=-=-=
%	FRAME: First Slide
%-=-=-=-=-=-=-=-=-=-=-=-=-=-=-=-=-=-=-=-=-=-=-=-=
\begin{frame}
\frametitle{Outline}
  \tableofcontents
\end{frame}

\begin{frame}
\frametitle{Intro}
  Was ist überhaupt Bootstrapping?
  
  \begin{quote}"In the field of electronics, a technique where part of the output of a system is used at startup can be described as bootstrapping."\footcite{https://en.wikipedia.org/wiki/Bootstrapping_(electronics)}
  \end{quote}
\end{frame}
%\section{Problem description}
%\begin{frame}[fragile]
%\frametitle{Non-line-of-sight (NLOS)}
%\begin{columns}
%\begin{column}{0.5\textwidth}
%		\begin{block}{Error sources by significance}
%   
%\begin{enumerate}
%\item Multipath
%\item Signal level dependent receive delay as described in \autocite{DW1000APS11}
%\item Noise
%\item Transmission and receive delays
%\item Clock drift between anchor and tag 
%\end{enumerate}
%\end{block}
%\end{column}
%\begin{column}{0.5\textwidth}  %%<--- here
%    \begin{center}
%     \begin{figure}
%		\graphicspath{{fig/}}
%		\def\svgwidth{\textwidth}
%		\input{fig/NLOS_Problem.pdf_tex}
%		\caption{Typical NLOS scenario} 
%		\label{fig:measurementSetup}
%\end{figure}
%     \end{center}
%     
%\end{column}
%\end{columns}
%\end{frame}
%
%\begin{frame}
%\frametitle{LOS vs. NLOS}
%\begin{columns}
%\begin{column}{0.5\textwidth}
%\begin{figure}
%	\input{fig/Bad_CIR.tex}
%	\caption{Misdetected first path due to obstruction.} 
%	\label{fig:LOS_NLOS_histogram}
%\end{figure}
%\end{column}
%\vspace{0 pt}
%\begin{column}{0.5\textwidth}  %%<--- here
%\begin{figure}
%	\input{fig/Good_CIR.tex}
%	\caption{Correct detection of first path in good conditions.} 
%	\label{fig:LOS_NLOS_histogram}
%\end{figure}
%\end{column}
%\end{columns}
%\end{frame}
%
%\section{Approach}
%\begin{frame}
%\frametitle{Our approach}
%\begin{alertblock}{Independent observations}
%	The goal is to gather more information i.e. collect and process independent measurements to get a better range/location estimate.
%	
%	
%	Time averaging typically yields low variance but \textbf{biased} location estimate due to multipath.
%\end{alertblock}
%\begin{block}{A possible approach}
%\vspace{6px}
%\begin{columns}[T]
%    \begin{column}{.6\linewidth}
%    Use multiple antenna elements to gain Angle of Arrival (AoA) information which can 			be used for an improved location solution.
% %   \rule{\columnwidth}{\columnwidth}
%    \end{column}
%    \begin{column}{.29\linewidth}
%		\includegraphics[scale=0.07]{fig/AoA_Method.png}
%    \end{column}
%\end{columns}
%%\begin{columns}[T]
%%	\begin{column}[b]{0.5\textwidth}
%%		Use multiple antenna elements to gain Angle of Arrival (AoA) information which can 			be used for an improved location solution.
%%	\end{column}
%%	\begin{column}{0.5\textwidth}
%%		\includegraphics[scale=0.1]{fig/AoA_Method.png}
%%	\end{column}
%%\end{columns}
%\end{block} 
%\end{frame}
%
%\begin{frame}
%\frametitle{Beamforming and AoA estimation}
%\begin{columns}[T]
%\begin{column}{0.5\textwidth}
%	\begin{block}{Delay-sum beamforming}
%		\begin{equation}\label{equ:delaySumBF}
%   		Y(t;\bm{k}) = \sum^{N-1}_{n=0}X_n(t)e^{-j\bm{k}^T\bm{r}_n}
%		\end{equation}
%	\end{block}
%	\begin{block}{AoA estimation}
%		\begin{equation}\label{equ:AoAOptimization}
%       		\hat{\bm{k}}=\underset{\bm{k}}{\argmax} \left|Y(t;\bm{k})\right|
%		\end{equation}
%		\begin{equation}
%			\hat{\alpha} = f(\hat{\bm{k}})
%		\end{equation}
%		using
%		\begin{equation}\label{equ:LinearSpacing}
%    \bm{r}_n = 
%        \begin{bmatrix}
%           nd \\
%           0 \\
%           0
%         \end{bmatrix}\,, \quad
%    \bm{k} = 
%        \begin{bmatrix}
%           \sin\alpha \\
%           -\cos\alpha \\
%           0
%         \end{bmatrix}\frac{2\pi}{\lambda}\,.
%		\end{equation}
%	\end{block}
%\end{column}
%\begin{column}{0.5\textwidth}
%	\begin{figure}
%    		\centering
%    		\input{fig/PlaneWave.tex}
%    		\caption{Narrowband plane wave incident on a uniform linear antenna array.}
%    		\label{fig:planeWave}
%	\end{figure}
%\end{column}
%\end{columns}
%\end{frame}
%
%\begin{frame}
%\frametitle{Possible hardware implementations}
%\begin{columns}[T,onlytextwidth]
%	\begin{column}{0.33\textwidth}
%		\begin{block}{Analog}
%			\begin{figure}
%				\centering
%				\includegraphics[width = \textwidth]{fig/analog_BF.png}
%			\end{figure}
%		\end{block}
%	\end{column}
%	\begin{column}{0.33\textwidth}
%		\begin{block}{Digital}
%			\begin{figure}
%				\centering
%				\includegraphics[width = \textwidth]{fig/digital_BF.png}
%			\end{figure}
%		\end{block}
%	\end{column}
%	\begin{column}{0.33\textwidth}
%		\begin{block}{Hybrid}
%			\begin{figure}
%				\centering
%				\includegraphics[width = \textwidth]{fig/hybrid_BF.png}
%			\end{figure}
%		\end{block}
%	\end{column}
%\end{columns}
%\end{frame}
%
%\begin{frame}
%\frametitle{UWB AoA (PDOA) implementations/previous work}
%\begin{columns}[T,onlytextwidth]
%
%	\begin{column}{0.5\textwidth}
%	Recent UWB AoA papers:
%	\begin{itemize}
%		\item AnguLoc, Heydariaan et al.
%		\item ULoc, Zhao et al.
%		\item Angle of arrival estimation using decawave DW1000 integrated circuits, Dotlic et al.
%		\item Single-Antenna AoA Estimation with UWB Radios, Smaoui et al.
%	\end{itemize}
%	All implementations use are hardware expensive and need one transceiver per antenna.
%	\end{column}
%	\hfill
%	\begin{column}{0.4\textwidth}
%		\begin{figure}
%			\centering
%			\includegraphics[width = \textwidth]{fig/MultipleRecieveChains.png}
%			\caption{UWB PDOA Kit containing two DW1000 UWB transceivers clocked by the same 		 oscillator.}
%
%		\end{figure}
%	\end{column}
%\end{columns}
%\end{frame}
%
%
%
%\begin{frame}
%\frametitle{Our Approach}
%\begin{columns}[T]
%\begin{column}{0.55\textwidth}
%	\begin{alertblock}{Time multiplexing}
%		Send multiple consecutive packets at the \textbf{same} antenna and receive at \textbf{different} antennas at the tag.
%		
%		$\Rightarrow$ only one transceiver per tag.
%	\end{alertblock}
%	\begin{figure}
%    \centering
%    		\input{fig/ExperimentalSetup.tex}
%    		\caption{Experimental setup to demonstrate the time multiplexed AoA scheme.}
%    \label{fig:ExpSetup}
%\end{figure}
%
%\end{column}
%\begin{column}{0.45\textwidth}
%\begin{figure}
%         \centering
%\input{fig/ProposedScheme.tex}
%    \caption{Transmissions between nodes illustrated over time for the proposed ranging scheme.}
%    \label{fig:RangingScheme}
%\end{figure}
%\end{column}
%\end{columns}
%\end{frame}
%
%\begin{frame}
%\frametitle{The importance of coherence}
%\begin{columns}[T]
%	\begin{column}{0.6\textwidth}
%		\begin{block}{From phase to time}
%		The measured times
%			\begin{equation}
%				t = c_\phi\,p(t_0)\quad \tau = c_\phi\,\varphi(t_0)
%			\end{equation}
%			are proportional to the phases of the oscillators.
%			The phase difference between anchor and tag after a timespan $\Delta t_0$ is
%			\begin{equation}
%			\Delta \phi = p(\Delta t_0)-\varphi(\Delta t_0)\,.
%			\end{equation}
%		\end{block}
%	\end{column}
%	\begin{column}{0.4\textwidth}
%		\begin{figure}
%    			\centering
%    			\input{fig/TimeStretching.tex}
%    			\caption{Tag time $\tau$, true time $t_0$ and anchor time $t$ illustrated on number lines. }
%    			\label{fig:timeStretching}
%		\end{figure}
%	\end{column}
%\end{columns}
%\end{frame}
%
%\begin{frame}
%\frametitle{The importance of coherence: Example}
%\begin{columns}[T]
%	\begin{column}{0.6\textwidth}
%		\begin{block}{Example}
%
%			10ppm frequency error, $f_c=\SI{4.5}{GHz}$, $\Delta t_0 = \SI{5}{\micro\second}$
%			gives
%			\begin{equation}
%			\Delta \phi = 2\pi f_c\Delta t\frac{|\Delta f|}{f} =2\pi\cdot 0.45\,.
%			\end{equation}
%			In reality we have $\Delta t_0 = \SI{2500}{\micro\second}$ thus
%			\begin{equation}
%			\Delta \phi = 2\pi f_c\Delta t\frac{|\Delta f|}{f} =2\pi \cdot 112.5\,.
%			\end{equation}			
%		\end{block}
%	\end{column}
%	\begin{column}{0.4\textwidth}
%		\begin{figure}
%    			\centering
%    			\vspace{-0.5cm}
%    			\input{fig/Stopwatch.tex}
%    			\caption{Stopwatch analogy.}
%    			\label{fig:timeStretching}
%		\end{figure}
%	\end{column}
%\end{columns}
%\end{frame}
%
%
%\begin{frame}
%\frametitle{Clock model and clock drift compensation}
%\begin{columns}[T]
%	\begin{column}{0.5\textwidth}
%		\begin{block}{Clock model}
%			The mapping between anchor time/phase $t$/$p$ and tag time/phase $\tau$/$\varphi$ is modelled by an affine transformation
%			\begin{equation}
%			\tau(t) = a\,t+b\,,\quad \varphi(t) = a\,p+b_\varphi\,.
%			\end{equation}	
%		\end{block}
%		\only<2->{But how to get clock parameters $a$ and $b$?}
%	\end{column}
%	\begin{column}{0.5\textwidth}
%		\begin{block}{Clock drift compensation}
%			Received phase at tag is composed of
%			\begin{align}
%    			\varphi^\text{RX}_n &= \varphi^\text{TOF} + \varphi_n^\text{AoA} +\varphi_n^\text{TX}\, 				\nonumber\\
%   			 &= \varphi^\text{TOF} + \varphi_n^\text{AoA} + a\,p_n^\text{TX}+b_\varphi \, .						\label{equ:receivedPhase}
%			\end{align}
%			Thus the phase proportional to the AoA is
%				\begin{equation}\label{equ:phasediffAoA}
% 				   \varphi_n^\text{AoA} = \varphi^\text{RX}_n \underbrace{- a\,p_n^\text{TX}-b_\varphi -	\varphi^\text{TOF}}_{\text{compensation}}
%				\end{equation}
%		\end{block}
%	\end{column}
%\end{columns}
%\end{frame}
%
%\begin{frame}
%\frametitle{Clock parameter estimation}
%\begin{columns}[T]
%	\begin{column}{0.6\textwidth}
%		\begin{block}{Linear model}
%			\only<1>{
%			The receive timestamps for request and response 0 are
%			\begin{align}\label{equ:ClockModelTOF}
% 		 \textcolor{red}{\tau^{\text{RX}(m)}_{\text{Req}}} &= \tau^{\text{TX}(m)}_{\text{Req}}+\tau^{\text{TOF}(m)}			_\text{Req}\,,\\ 
% 		 \tau^{\text{RX}(m)}_{0} &= \textcolor{red}{\tau^\text{TX}_{0}}+\tau^{\text{TOF}(m)}_0\,.
%			\end{align}}	
%			\only<2-3>{
%			The receive timestamps for request and response 0 are
%			\begin{align}
%  				\textcolor{red}{a\, t^{\text{RX}(m)}_{\text{Req}} + b} &= \tau^{\text{TX}(m)}_{\text{Req}}+								\tau^{\text{TOF}(m)}_\text{Req}\label{equ:clockequ1}\,,\\ 
%  				\tau^{\text{RX}(m)}_{0} &= \textcolor{red}{a\,t^{\text{TX}(m)}_{0}+b}+ \tau^{\text{TOF}(m)}_{0}\label{equ:clockequ2}\,.						
%			\end{align}}
%			\only<3-4>{
%			Subtracting (12) from (13) gives
%			\begin{equation}
%    \tau^{\text{TX}(m)}_{\text{Req}}+\tau^{\text{RX}(m)}_0 = a\,(t^{\text{RX}(m)}_{\text{Req}}+t^{\text{TX}(m)}_0)+2b
%			\end{equation}
%			}
%			\only<4->{
%				Applied to $M$ ranging cycles yields a linear model
%				\begin{scriptsize}
%				\begin{equation}\label{equ:linEstModel}
%    \underbrace{
%    \begin{bmatrix}
%        \tau^{\text{TX}(0)}_{\text{Req}}+\tau^{\text{RX}(0)}_0\\
%        \vdots\\
%        \tau^{\text{TX}(M-1)}_{\text{Req}}+\tau^{\text{RX}(M-1)}_0\\        
%    \end{bmatrix}}_{\bm{y}}=
%    \underbrace{
%    \begin{bmatrix}
%        t^{\text{RX}(0)}_{\text{Req}}+t^{\text{TX}(0)}_0 & 2\\
%        \vdots   &  \vdots\\ 
%        t^{\text{RX}(M-1)}_{\text{Req}}+t^{\text{TX}(M-1)}_0 & 2\\
%    \end{bmatrix}}_{\bm{H}}
%    \underbrace{
%    \begin{bmatrix}
%        a\\
%        b\\
%    \end{bmatrix}}_{\bm{\theta}}
%				\end{equation}
%				\end{scriptsize}
%				with the LS-solution $\hat{\bm{\theta}} = (\bm{H}^T\bm{H})^{-1}\bm{H}\bm{y}$.
%			}
%		\end{block}
%		
%	\end{column}
%	\begin{column}{0.4\textwidth}
%		\begin{figure}
%         	\centering
%			\input{fig/ProposedScheme.tex}
%    			\caption{Transmissions between nodes illustrated over time for the proposed ranging scheme.}
%    			\label{fig:RangingScheme}
%		\end{figure}
%	\end{column}
%\end{columns}
%\end{frame}
%
%
%\begin{frame}
%\frametitle{Summary}
%\begin{columns}[T]
%	\begin{column}{0.5\textwidth}
%	\begin{block}{AoA Estimation procedure}
%		\begin{enumerate}
%		\item Estimate clock parameters based on last $M$ ranging cycles.
%		\item Get CIR at the reported first path index $X_{n,\text{FP}}$
%		\item Compensate phase of $X_{n,\text{FP}}$
%		\item Estimate AoA $\hat{\alpha}$
%		\end{enumerate}
%	\end{block}
%		\begin{figure}
%			\input{fig/CyclewiseClockEst.tex}
%			\vspace{-0.2 cm}
%			\caption{Windowed clock estimation.}
%		\end{figure}
%	\end{column}
%	\begin{column}{0.5\textwidth}
%		\begin{figure}
%			\input{fig/Blockdiagram.tex}
%			\caption{AoA estimation procedure.}
%			
%		\end{figure}
%	\end{column}
%\end{columns}
%\end{frame}
%
%
%
%\section{Experiments}
%\begin{frame}
%\frametitle{Experimental setup}
%\begin{columns}[T]
%	\begin{column}{0.6\textwidth}
%		\begin{block}{Hardware}
%		\begin{itemize}
%		
%		
%		\item Qorvo DW1000 UWB tranceiver:
%			\begin{itemize}
%			\item IEEE802.15.4-2011 UWB compliant
%			\item Supports 6 RF bands from 3.5 GHz to 6.5 GHz
%			\item Data rates of 110 kbps, 850kbps, 6.8 Mbps
%			\item 1 GS/s Baseband sampling rate
%			\end{itemize}
%		\item Antenna switch (SP4T)
%		\item Uniform linear array with a spacing of $\lambda/2$
%		\end{itemize}	
%		\end{block}
%		\begin{block}{Experiments}
%			\begin{itemize}
%				\item Reference measurement
%				\item Turntable measurement
%				\item Joint ranging and AoA
%			\end{itemize}
%		\end{block}
%
%	\end{column}
%	\begin{column}{0.4\textwidth}
%		\begin{figure}
%   			 \centering
%   			 \input{fig/RefMeasurement.tex}
%  			  \caption{Measurement setup of reference measurement.}
%  			 \label{fig:RefMeasurement}
%		\end{figure}
%		\vspace{-1em}
%		\begin{figure}
% 		   \centering
%		    \input{fig/TurntableMeasurement.tex}
%		    \caption{Measurement setup of turntable measurement.}
% 		   \label{fig:turnTableMeasurement}
%		\end{figure}
%	\end{column}
%\end{columns}
%\end{frame}
%
%\begin{frame}
%\frametitle{Results - Reference measurement}
%\begin{columns}[T]
%	\begin{column}{0.5\textwidth}
%	\begin{figure}
%   \hspace{1.3cm}
%    \input{fig/minimumPhaseErrorHist.tex}
%    \caption{Distribution of phase errors of the CIRs received at the same antenna. The optimum of $M=9$ cycles was used for clock estimation.}
%    \label{fig:minimumPhaseErrorHist}
%\end{figure}
%	\end{column}
%	\begin{column}{0.5\textwidth}
%	\begin{figure}
%    \centering
%    \input{fig/RMSE_over_cycles.tex}
%    \caption{RMSE of phase error plotted over the number of cycles $M$ used for clock estimation. One cycle takes $\SI{14}{ms}$ to complete.}
%    \label{fig:RMSEOverCycles}
%\end{figure}
%	\end{column}
%\end{columns}
%\end{frame}
%
%\begin{frame}
%\frametitle{Results - Turntable measurement}
%\begin{columns}[T]
%	\begin{column}{0.5\textwidth}
%\begin{figure}
%    \centering
%    \input{fig/estAoAvsTrue.tex}
%    \caption{Estimated AoA $\hat{\alpha}$ evaluated at reported first path index versus true angle of arrival $\alpha$ obtained from the rotary tables angular position. $M = 20$ cycles are used for clock estimation.}
%    \label{fig:EstAoAvsTrue}
%\end{figure}
%	\end{column}
%	\begin{column}{0.5\textwidth}
%	\begin{figure}
%    \centering
%    \input{fig/ErrorCDF.tex}
%    \caption{Cumulative distribution function of absolute AoA error of the rotary table measurement. The error was computed for all true angles $\alpha \in [\SI{-45}{\degree},\SI{-45}{\degree}]$.}
%    \label{fig:ErrorCDF}
%    \end{figure}
%	\end{column}
%\end{columns}
%\end{frame}
%
%\begin{frame}
%\frametitle{Results - Joint ranging and AoA}
%\begin{columns}[T,onlytextwidth]
%	\begin{column}{0.5\textwidth}
%	\hspace{1.3cm}
%		\begin{figure}
%    			\input{fig/trainTrack.tex}
%    			\label{fig:trainTrack}
%		\end{figure}
%	\end{column}
%	\begin{column}{0.5\textwidth}
%	\begin{figure}
%		\caption{Combined range and AoA estimation of the moving anchor mounted on a toy train.}
%	\end{figure}
%	\end{column}
%\end{columns}
%\end{frame}
%
%\begin{frame}
%\frametitle{Conclusions}
%\begin{columns}[T,onlytextwidth]
%	\begin{column}{0.49\textwidth}
%		\begin{block}{Conclusions}
%			\begin{itemize}
%				\item Proof of concept showed promising results.
%				\item Accuracy similar to other works \autocite{ruiz2017comparing}\autocite{AoAQorvo}
%				\item Additional AoA information
%				\item $N-1$ reduction in transceiver ICs compared to $N$ parallel receive chains
%			\end{itemize}
%		\end{block}
%	\end{column}
%	\begin{column}{0.49\textwidth}
%		\begin{block}{Future Work}
%		\begin{itemize}
%			\item Improve clock estimation
%			\item Full scale joint ranging and AoA localization		
%		\end{itemize}
%		\end{block}
%	\end{column}
%\end{columns}
%\end{frame}
%
%
%\section{Backup}
%\begin{frame}
%\frametitle{DW1000 Reciever}
%\begin{figure}
%\centering
%\includegraphics[height = 0.8\textheight]{fig/DW1000Blockdiagram.png}
%\end{figure}
%\end{frame}
%
%\begin{frame}
%\frametitle{Beamforming}
%\begin{figure}
%\centering
%\includegraphics[height = 0.8\textheight]{fig/BeamForming_ArrayAntenna_Backup.png}
%\end{figure}
%\end{frame}
%
%\begin{frame}
%\frametitle{2D Localization}
%\begin{figure}
%\centering
%\includegraphics[height = 0.8\textheight]{fig/TrainTrackSetup.jpg}
%\end{figure}
%\end{frame}
%
%
%\begin{frame}
%\frametitle{DFT based AoA}
%\begin{columns}[T,onlytextwidth]
%	\begin{column}{0.49\textwidth}
%		\begin{block}{}
%\begin{equation}\label{equ:delaySumBF}
%  				 Y(t;\bm{k}) = \sum^{N-1}_{n=0}X_n(t)e^{-j\bm{k}^T\bm{r}_n}\,.
%			\end{equation}
%			\begin{equation}
%   				 \bm{k} = -\frac{2\pi}{\lambda}\bm{u}
%			\end{equation} 
%	\begin{equation}\label{equ:LinearSpacing}
%	    \bm{r}_n = 
%	        \begin{bmatrix}
%	           nd \\
%	           0 \\
%	           0
%	         \end{bmatrix}\,, \quad
%	    \bm{k} = 
%	        \begin{bmatrix}
%	           \sin\alpha \\
%	           -\cos\alpha \\
%	           0
%	         \end{bmatrix}\frac{2\pi}{\lambda}\,.
%	\end{equation}
%	\begin{equation}\label{equ:psiDef}
%	    \psi(\alpha):= \frac{d\sin\alpha}{\lambda}\,,
%	\end{equation}
%		\end{block}
%	\end{column}
%	\begin{column}{0.49\textwidth}
%		\begin{block}{}
%		\begin{align}
%		   Y_k(\psi) &= \sum^{N-1}_{n=0}X_{n,k}e^{-j\,n\,2\pi\,\psi}\,.\label{equ:delaySumBFLinear}
%		\end{align}
%		\begin{equation}\label{equ:AoAOptimization}
%		       \hat{\psi}=\underset{\psi}{\argmax} \left|\sum^{N-1}_{n=0}X_{n,\text{FP}}\, e^{-jn\,2\pi\,\psi}\right|\,.
%		\end{equation}
%		\end{block}
%	\end{column}
%\end{columns}
%\end{frame}
\end{document}
			